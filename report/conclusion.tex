\section{Conclusion} \label{sec:conclusion}
%
In this project, we aim to learn and have some insights about GraphX, a graph processing framework built on top of Spark.
%
We learn GraphX by leveraging it to easily accomplish paper recommendation at scale.
%
In GraphX, we implement PageRank algorithm without using the provided PageRank API and apply it on a citation network.
%
This would help us rank the importance of the papers.
%
Besides direct execution of PageRank, we allow users to provide keywords to narrow the search space.
%
Users can also choose to provide their reading history.
%
Based on the history, we derive a list of papers that the users might be interested by pattern finding.
%
Through the project, we have gained a number of insights and understood the advantages and disadvantages of GraphX.

Future work:
According to the previous analysis, we could improve the system in several ways. First we can include external TF/IDF score or compute the score according to the whole paper to eliminate problem from the language unicity. In this way, we can acquire better keywords. Second, we can try other valid citation network dataset for more updated data. Also, we can explore other pattern mining method and try different recommendation algorithms. Another very important work should be done is that we have to solve the cold start problem and develop auto evaluation process for unlabeled recommendation problem. 
