\section{Framework}
%
In this section, we discuss about the big data processing framework for graphs, GraphX, that we use to tackle our problem.
%
\subsection{GraphX}
%
We highlight some of the features in GraphX and discuss about the advantages and disadvantages.
%
\subsubsection{Highlights and Pregel API}
%
GraphX is a component built on top of Spark therefore has adopted many benefits and features from Spark.
%
In Spark, data to be processed is represented as Resilient Distributed Dataset (RDD), which is immutable and distributed.
%
In GraphX, graph data is abstracted as a directed graph allowing properties associated with vertices and edges.
%
GraphX extends Spark RDD to represent a graph.
%
In particular, a graph is represented as VertexRDD and EdgeRDD.
%
Since VertexRDD and EdgeRDD are they are simply extensions of RDD, operations that can be applied on RDD can also be applied on them.
%

In terms of the richness of operators, GraphX provides a set of fundamental operators such as "subgraph," "reverse edges," etc. as well as a few common graph algorithms such as PageRank, ConnectedComponents, etc..
%
One feature that we use heavily in GraphX is the Pregel API.
%
Pregel API uses the concept of "thinking like a vertex."
%
In other words, the API provides a vertex-centric way of processing graphs.
%
Users are required to specify the following functions for a Pregel API call.
%
\begin{itemize}
%
\item \textbf{Vertex Program} (denoted as $vprog$): 
%
This function runs on every vertex of the graph.
%
In this function, each vertex $V$ has the access of the aggreated messsage (see below) delivered to $V$ as well as the attribute of $V$.
%
The function returns the new state (i.e. attribute) of the vertex.
%
\item \textbf{Send Message Program} (denoted as $sendMsg$):
%
This function runs on the edges in the EdgeRDD. 
%
The input of this function is an edge triplet, which is a view logically joining the vertex and edge properties
%
The function returns a message to be sent to the neighbors of each vertex.
%
\item \textbf{Merge Message Program} (denoted as $mergeMsg$):
%
This function combines two messages that are sent to the same vertex in an iteration into one.
%
The function returns the combined message.
%
Because of this aggregation, when a vertex executes the vertex program in the next iteration, only the aggreated message is passed in.
%
\end{itemize}
%
In addition, users need to specify the following parameters.
%
\begin{itemize}
%
\item \textbf{Initial Message} (denoted as $initialMsg$):
%
This is the message passed to every vertex in the first iteration.
%
The purpose of this message is to make sure each vertex executes the vertex program at least once.
%
\item \textbf{Max iteration} (denoted as $maxIterations$):
%
This specifies the maximum number of iterations that can be run for a Pregel API call.
%
\item \textbf{Edge direction} (denoted as $activeDirection$):
%
This specifies the direction of the edges on which the $sendMsg$ function will run.
%
\end{itemize}
%
As a summary of how Pregel works, there are two major parts of the computation in each iteration.
%
\begin{enumerate}
%
\item \textbf{Execution of Vertex Program}: 
%
On each vertex, the vertex program would be run sequentially.
%
Note that only those vertices that have incoming messages from previous iteration will be run.
%
\item \textbf{Message Preparation}:
%
A message that will be sent would be constructed and delivered at the beginning of next iteration.
%
\end{enumerate}


\subsubsection{Advantages and Disadvantages}
- unlike standard Pregel, it limits message passing to neighbors (discuss good and bad about this)
- If the graph is evolving (new vertex or edges are being added) currently the graph needs to be recreated and hence it is necessary to do checkpointing else in case of node failure recovery time will be a lot.

