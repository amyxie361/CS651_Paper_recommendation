\section{GraphX}
%
In this section, we discuss about the big data processing framework for graphs, GraphX, that we use to tackle our problem.
%
We will highlight some of the features in GraphX and go through some of the essential details of Pregel API.
%
\subsection{Highlights}
%
GraphX is a component built on top of Spark therefore has adopted many benefits and features from Spark.
%
In Spark, data to be processed is represented as Resilient Distributed Dataset (RDD), which is immutable and distributed.
%
In GraphX, graph data is abstracted as a directed graph allowing properties associated with vertices and edges.
%
GraphX extends Spark RDD to represent a graph.
%
In particular, a graph is represented as VertexRDD and EdgeRDD.
%
Since VertexRDD and EdgeRDD are they are simply extensions of RDD, operations that can be applied on RDD can also be applied on them.
%
\subsection{Pregel API}
%
In terms of the richness of operators, GraphX provides a set of fundamental operators such as "subgraph," "reverse edges," etc. as well as a few common graph algorithms such as PageRank, ConnectedComponents, etc..
%
One feature that we use heavily in GraphX is the Pregel API.
%
Pregel API uses the concept of "thinking like a vertex."
%
In other words, the API provides a vertex-centric way of processing graphs.
%
Users are required to specify the following functions for a Pregel API call.
%
\begin{itemize}
%
\item \textbf{Vertex Program} (denoted as $vprog$): 
%
This function runs on every vertex of the graph.
%
In this function, each vertex $V$ has the access of the aggreated messsage (see below) delivered to $V$ as well as the attribute of $V$.
%
The function returns the new state (i.e. attribute) of the vertex.
%
\item \textbf{Send Message Program} (denoted as $sendMsg$):
%
This function runs on the edges in the EdgeRDD. 
%
The input of this function is an edge triplet, which is a view logically joining the vertex and edge properties
%
The function returns a message to be sent to the neighbors of each vertex.
%
\item \textbf{Merge Message Program} (denoted as $mergeMsg$):
%
This function combines two messages that are sent to the same vertex in an iteration into one.
%
The function returns the combined message.
%
Because of this aggregation, when a vertex executes the vertex program in the next iteration, only the aggreated message is passed in.
%
\end{itemize}
%
In addition, users need to specify the following parameters.
%
\begin{itemize}
%
\item \textbf{Initial Message} (denoted as $initialMsg$):
%
This is the message passed to every vertex in the first iteration.
%
The purpose of this message is to make sure each vertex executes the vertex program at least once.
%
\item \textbf{Max Iteration} (denoted as $maxIterations$):
%
This specifies the maximum number of iterations that can be run for a Pregel API call.
%
\item \textbf{Edge Direction} (denoted as $activeDirection$):
%
This specifies the direction of the edges on which the $sendMsg$ function will run.
%
\end{itemize}
%
As a summary of how Pregel works, there are two major parts of the computation in each iteration.
%
\begin{enumerate}
%
\item \textbf{Execution of Vertex Program}: 
%
On each vertex, the vertex program would be run sequentially.
%
Note that only those vertices that have incoming messages from previous iteration will be run.
%
\item \textbf{Message Preparation}:
%
A message that will be sent would be constructed and delivered at the beginning of next iteration.
%
\end{enumerate}
%
We express the execution of a Pregel API call at a high level in Algorithm \ref{alg:pregel}.
%
\begin{algorithm}
  \SetKwInOut{Input}{Input}
  \SetKwInOut{Output}{Output}
  \SetKwInOut{Init}{Init}
  \Input{Initial Graph: $G = (V, E)$}
  \Output{Updated Graph: $G = (V, E)$}
  \SetAlgoLined
  \SetKwProg{Fn}{Function}{:}{end}
  \Fn{Pregel(initialMsg, maxIterations, activeDirection, vprog, sendMsg, mergeMsg)}{
     numIterations := 0 \;
     initialMsg is sent to each vertex \;
     \While{\# of vertices that will receive msg > 0 OR numIterations < maxIterations}{
       vprog(...) // run in parallel on each vertex \;
       sendMsg(...) // run in parallel on each edge satisfying activeDirection \;
       mergeMsg(...) // run in parallel on each set of messages sent to the same vertex \;
     }
  }
  \caption{Pregel API}
  \label{alg:pregel}
\end{algorithm}
%

