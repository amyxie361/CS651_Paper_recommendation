\section{Introduction}

With the rapid publication of scientific literature, conducting a comprehensive literature review has become more challenging. Keeping up with current development of a certain also requires huge effort. However, citation network would help to improve the efficiency and quality when we want to survey on a certain field. Since the citation network is a graph structure, we will explore the new framework \textbf{GraphX}, which makes graph computation easier. The objective of this project is to learn a new framework GraphX and implement network algorithms to accomplish \textbf{paper recommendation}. Our system will recommend users related papers based on input keywords, citation network and reading history. 

We will solve the problem of paper recommendation based on keywords, citation network and user reading history. 
We will have an offline dataset consisting of paper citation relation and papers' content.

For each query, we have the following as the input and output.

\paragraph{Input}
\begin{itemize}
  \item \textbf{Keywords}: a list of strings where each string is an interested field, e.g. ["machine learning", "computer vision"].
  \item \textbf{Reading history (optional)}: a list of papers where each paper is represented as a unique id, e.g. ["journals/cacm/Szalay08"].
\end{itemize}

\paragraph{Output}
\begin{itemize}
  \item \textbf{Recommended papers}: a list of papers in descending recommending order.
\end{itemize}

