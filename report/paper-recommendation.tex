\documentclass[sigconf]{acmart}

\usepackage{booktabs} % For formal tables
\usepackage{amsmath}
\usepackage{graphicx}
\usepackage[]{algorithm2e}
\graphicspath{ {./} }
\usepackage{listings}

\lstset{
  columns=fullflexible,
  frame=single,
  breaklines=true,
  postbreak=\mbox{\textcolor{red}{$\hookrightarrow$}\space},
}

% Copyright
\setcopyright{none}
%\setcopyright{acmcopyright}
%\setcopyright{acmlicensed}
%\setcopyright{rightsretained}
%\setcopyright{usgov}
%\setcopyright{usgovmixed}
%\setcopyright{cagov}
%\setcopyright{cagovmixed}

\settopmatter{printacmref=false, printccs=false, printfolios=false}

\begin{document}
\title{Paper Recommendation using GraphX}

\author{Jeremy Chen}
\affiliation{%
  \institution{University of Waterloo}
}
\email{jeremy.chen@uwaterloo.ca}

\author{Junyi Zhang}
\affiliation{%
  \institution{University of Waterloo}
}
\email{j823zhan@uwaterloo.ca}

\author{Yuqing Xie}
\affiliation{%
  \institution{University of Waterloo} }
\email{yuqing.xie@uwaterloo.ca}

% removes footnote with conference information in first column
\renewcommand\footnotetextcopyrightpermission[1]{}
% removes running headers
\pagestyle{plain}


\begin{abstract}
Data in graph structure has become popular nowadays.
%
Graph processing framework, therefore, becomes an essential tool to perform scalable computation.
%
GraphX is one of the top choices because of its easiness, scalability, and performance.
%
In this project, we explore GraphX in a deeper way and gain some insights.
%
We achieve this by using GraphX to build an academic paper recommendation system.
%
Without using the trivial way, we implement PageRank, keyword filtering, and pattern finding algorithms in GraphX.
%
We apply them on a citation network to give recommendations of papers, with taking users' interest into account.
%
We have two metrics to evaluate our recommendations: inversion count and ranking distance, and both have given us positive feedback.
%
Through doing this project, we have not only studied GraphX in a practical way but performed data analytics on a citation network.
%
\end{abstract}

\maketitle

\section{Introduction}
%
With the rapid publication of scientific literature, conducting a comprehensive literature review has become more challenging.
Hundreds of papers in computer science are published every day.
%
Keeping up with current development of a certain area also requires huge effort.
%
With limited time and energy, researchers would like to invest their time on the papers that can give them the most inspiration.
%
However, without any assistance on filtering papers, finding representable papers in a particular field seems impossible.

%
More and more information retrieval method and graph processing method are developed during the past few years. 
What if we could combine the ability of distributed system with the advanced graph processing algorithms together with the text processing techniques?
If there is a system that can automatically extract interesting or classic papers to researchers, we can save more time from only manually filtering the papers that we are interested in, based on the title and the abstracts.
In this project, we proposed a pipeline that could take interested keywords or a list of past read papers as the input and then recommend related papers that the user will be interested in.
We combined the crawled abstracts of papers with the citation network to help the paper recommendation. We learnt and explore the new framework \textbf{GraphX}, which makes graph computation easy and scalable.
Since a citation network is in graph structure, we can leverage existing graph data processing framework to help us analyze the importance of each paper.

Our system is able to give interesting results based on both keywords mode and reading history mode. The recommended papers are most classic ones. This could help the researchers to trace back to the most important moment in the history, which is valuable both for new researchers and experienced researchers who need an exhausted literature survey.

The rest of the paper is structured as follows. 
%
Section \ref{sec:related-work} reviews related work on recommendation systems. 
%
We define the problem that we need to tackle in Section \ref{sec:problem} and also give the formal notation that we need.
%
Section \ref{sec:framework} discusses the framework we explored (i.e. GraphX), and we describe the algorithms we use in GraphX in Section \ref{sec:algo}.
%
In Section \ref{sec:insights}, we analyze the advantages and disadvantages of GraphX and discuss what we have learned while exploring the framework.
%
We present our experimental results in Section \ref{sec:exp} and further analyze them.
%
Section \ref{sec:conclusion} states some future work of this project and concludes the paper.

\section{Related Work}

There are number of researches on recommendation system, and the methods generally could be divided into machine learning and graph. Our approach is based on graph, one relevant work would be 
a service that provides real-time recommendations to tens of millions of mobile users in Twitter by implementing real-time motif detection on large dynamic graphs ~\cite{Gupta:2014}. In addition, similarity measures ~\cite{Goel2013} have been extensively studied in the area of information retrieval and networks, i.e. user similarity computation and the effects of user similarity in social media. Similarities between two entities are used for community detection, similarity search and recommendations, and most of them proposed for networks are based on graph structure, i.e. SimRank ~\cite{Jeh:2002}, Penetrating Rank ~\cite{Zhao:2009}. 

The papers mentioned above and our project are all based on large-scale graph processing. One suitable system for this kind of problem is Pregel ~\cite{Malewicz:2010}, where programs are expressed as a sequence of iterations. In each iteration, a vertex could receive the message sent from the previous iteration, then it sends messages to other vertices, updates its own state and outgoing edges. Since large-scale graphs that with millions of vertices and billions of edges are difficult to analyze, researchers have usually turned to distributed solutions, i.e. MapReduce ~\cite{Lin:2010}, or reduce the size of the graph by partitioning \ref{Karypis:1998}.

Furthermore, there is a pretty related work to our project ~\cite{DBLP}, which is about content-based citation recommendation. However, the method used in this paper is neural model instead of graph. They first embed all available documents into a vector space, and encoded text content of each document. Then, chose the nearest neighbours of the query document as candidates and reranked the candidates by another model trained, where this model takes a pair of documents as input and estimates the probability that document2 should be cited in document1.

The evaluation methods ~\cite{Beel:2013} for recommendation system include online evaluations and offline evaluations. Online evaluation means that recommendations are shown to users as they use the real-world system, while offline evaluation ~\cite{Ricci:2010} includes precision, recall and F-measure. A good recommender system contributes to three features, recommendation accuracy, user satisfaction, and provider satisfaction, and leads to the question how these three features are to be quantified and compared.
\section{Problem Description}
%
We will solve the problem of paper recommendation based on keywords, citation network and user reading history. 
We will have an offline dataset consisting of paper citation relation and papers' content.

For each query, we have the following as the input and output.
%
\paragraph{Input}
%
\begin{itemize}
%
  \item \textbf{Keywords}: a list of strings where each string is an interested field, e.g. ["machine learning", "computer vision"].
%
  \item \textbf{Reading history (optional)}: a list of papers where each paper is represented as a unique id, e.g. ["journals/cacm/Szalay08"].
%
\end{itemize}
%
\paragraph{Output}
%
\begin{itemize}
  \item \textbf{Recommended papers}: a list of papers in descending recommending order.
\end{itemize}
%

\section{GraphX}
%
In this section, we discuss about the big data processing framework for graphs, GraphX, that we use to tackle our problem.
%
We will highlight some of the features in GraphX and go through some of the essential details of Pregel API.
%
\subsection{Highlights}
%
GraphX is a component built on top of Spark therefore has adopted many benefits and features from Spark.
%
In Spark, data to be processed is represented as Resilient Distributed Dataset (RDD), which is immutable and distributed.
%
In GraphX, graph data is abstracted as a directed graph allowing properties associated with vertices and edges.
%
GraphX extends Spark RDD to represent a graph.
%
In particular, a graph is represented as VertexRDD and EdgeRDD.
%
Since VertexRDD and EdgeRDD are they are simply extensions of RDD, operations that can be applied on RDD can also be applied on them.
%
\subsection{Pregel API}
%
In terms of the richness of operators, GraphX provides a set of fundamental operators such as "subgraph," "reverse edges," etc. as well as a few common graph algorithms such as PageRank, ConnectedComponents, etc..
%
One feature that we use heavily in GraphX is the Pregel API.
%
Pregel API uses the concept of "thinking like a vertex."
%
In other words, the API provides a vertex-centric way of processing graphs.
%
Users are required to specify the following functions for a Pregel API call.
%
\begin{itemize}

\item \textbf{Vertex Program} (denoted as $vProg$): 
%
This function runs on every vertex of the graph.
%
In this function, each vertex $V$ has the access of the aggreated messsage (see below) delivered to $V$ as well as the attribute of $V$.
%
The function returns the new state (i.e. attribute) of the vertex.
%
\item \textbf{Send Message Program} (denoted as $sendMsg$):
%
This function runs on the edges in the EdgeRDD. 
%
The input of this function is an edge triplet, which is a view logically joining the vertex and edge properties
%
The function returns a message to be sent to the neighbors of each vertex.
%
\item \textbf{Merge Message Program} (denoted as $mergeMsg$):
%
This function combines two messages that are sent to the same vertex in an iteration into one.
%
The function returns the combined message.
%
Because of this aggregation, when a vertex executes the vertex program in the next iteration, only the aggreated message is passed in.
%
\end{itemize}
%
In addition, users need to specify the following parameters.
%
\begin{itemize}
%
\item \textbf{Initial Message} (denoted as $initialMsg$):
%
This is the message passed to every vertex in the first iteration.
%
The purpose of this message is to make sure each vertex executes the vertex program at least once.
%
\item \textbf{Max Iteration} (denoted as $maxIterations$):
%
This specifies the maximum number of iterations that can be run for a Pregel API call.
%
\item \textbf{Edge Direction} (denoted as $activeDirection$):
%
This specifies the direction of the edges on which the $sendMsg$ function will run.
%
\end{itemize}
%
As a summary of how Pregel works, there are two major parts of the computation in each iteration.
%
\begin{enumerate}
%
\item \textbf{Execution of Vertex Program}: 
%
On each vertex, the vertex program would be run sequentially.
%
Note that only those vertices that have incoming messages from previous iteration will be run.
%
\item \textbf{Message Preparation}:
%
A message that will be sent would be constructed and delivered at the beginning of next iteration.
%
\end{enumerate}
%
We express the execution of a Pregel API call at a high level in Algorithm \ref{alg:pregel}.
%
\begin{algorithm}
  \SetKwInOut{Input}{Input}
  \SetKwInOut{Output}{Output}
  \SetKwInOut{Init}{Init}
  \Input{Initial Graph: $G = (V, E)$}
  \Output{Updated Graph: $G = (V, E)$}
  \SetAlgoLined
  \SetKwProg{Fn}{Function}{:}{end}
  \Fn{\textbf{Pregel}(initialMsg, maxIterations, activeDirection, vProg, sendMsg, mergeMsg)}{
     numIterations := 0 \;
     initialMsg is sent to each vertex \;
     \While{\# \{vertices that will receive msg > 0\} AND numIterations $\leq$ maxIterations}{
       \textbf{vProg}(Vertex, sumMsg) // run in parallel on each vertex \;
       \textbf{sendMsg}(Edge) // run in parallel on each edge satisfying activeDirection \;
       \textbf{mergeMsg}(aMsg, bMsg) // run in parallel on each set of messages sent to the same vertex \;
     }
  }
  \caption{Pregel API}
  \label{alg:pregel}
\end{algorithm}
%


\section{Algorithms}

For the separated two parts of programs, we develop our algorithm differently. 

\subsection{Keywords recommendation algorithm}

Before implement recommendation algorithm, we first implemented the TF/IDF algorithm to preprocessing papers' abstracts into their keywords. Then we implemented the classic PageRank algorithm using GraphX for recommendation score computation.

\subsubsection{Preprocessing}

For computation effectiveness, we only keep the abstract of each paper. We first tokenize the abstract, then implemented the TF/IDF algorithm. Now we acquired the TF/IDF score for each token in each paper. Then we filter $k$ tokens with the highest score in each paper to be the paper's keywords. Then we construct an inverse map from keywords to paper index, meaning the token is a keyword for those papers. 
%We will also take the original author-generated keywords into consideration.

\subsubsection{PageRank for recommendation}

After we get the keywords to papers map, we implement the classic page rank algorithm to compute the score. 

First, according to the keywords from the input, we filter out the subgraphs. Each paper in this subgraph have the target tokens as their keywords. 

Next, we will first use GraphX to construct the citation network.  After we constructed the graph, we will implement the classic PageRank algorithm to establish a paper ranking. {TODO: detailed algorithm in GraphX for Jeremy}

Then we give recommendation paper list according to the ranking score, from the highest to the lowest.

\subsubsection{Pattern mining for recommendation}

To take advantage of the reading history for recommendation, we use pattern mining algorithms. 

We will find common patterns such as diamond pattern and triangle pattern to do recommendation. To find the interested classic papers, we find patterns on the original graph. To find the interested recent paper, we find the patterns on the reversed graph.

To simplify the description, we first consider the original graph. The recent paper version will be the same after we turn the graph into a reversed version.

Because the restrictions from GraphX, we design the algorithm to be a two-iteration algorithm. And the operations of each iteration will be different. In the first iteration, we filter out the papers cited by the papers in the reading history. We call them referred papers. In the second iteration, we mine the patterns out. You can refer to the detailed algorithm in Algorithm \ref{alg:pattern}

\begin{algorithm}
 \SetKwInOut{Input}{Input}
 \SetKwInOut{Output}{Output}
 \SetKwInOut{Init}{Init}
 \Input{Paper list: $P = [p_1, p_2, ..., p_n]$}
 \Output{Recommended paper list: $R = [r_1, r_2, ..., r_m]$}
 %initial
\eIf{Vertex.Id $\in$ P}{Vertex.Attr $\gets$ 1.0}
	{Vertex.Attr $\gets$ 0.0}
Init.Message $\gets$ 0.0

 \While{not at end of this document}{
  read current\;
  \eIf{understand}{
   go to next section\;
   current section becomes this one\;
   }{
   go back to the beginning of current section\;
  }
 }
 \caption{How to write algorithms}
\end{algorithm}






\section{Insights}
%
Since the main component of this project consists of learning the framework GraphX, we list some insights and what we have learned whiling exploring it.
%
\subsection{Advantages of GraphX}
%
Since GraphX is built on top of Spark, it adopts a lot of advantages from Spark.
%
One of the biggest advantages of GraphX is the easiness of implementation.
%
As the purpose of this project is to learn the framework, we choose not to use the PageRank function provided by GraphX.
%
Instead, we implement the PageRank algorithm by ourselves using Pregel API in GraphX.
%
However, the setup in order to initiate the execution of PageRank using Pregel only takes approximately 20 lines of code to complete.
%
Then, GraphX handles everything else for us, including fault-tolerance, concurrency control, etc..
%

There is one significant difference between the implementation of Pregel API in GraphX and standard Pregel implementation.
%
The Pregel API in GraphX only allows sending messages to the neighbors, whereas standard Pregel implementation can send messages to any vertex in the graph.
%
Although this sounds like a disadvantage, this "limitation" helps GraphX improve the efficiency.
%
Since in GraphX a graph is an extension of RDD, a graph dataset would be distributed across multiple machines.
%
Intuitively, we could partition a graph based on the vertices, for example, by hashing vertex IDs.
%
However, this would cause large communication overhead as a lot of messages need to be sent to across machines, which becomes a bottleneck of the computation.
%
Instead, GraphX handles graph partitioning in a edge-based manner.
%
Edges are partitioned onto different machines and vertices are allowed to span on multiple machines.
%
This way, with the "limitation" that messages can only be sent to direct neighbors, we can partition the graph in a way that the communication overhead is very small.
%

\subsection{Disadvantages of GraphX}
%
While implementing algorithms in GraphX, we have also found some disadvantages of the framework.
%
First, we are not allowed to have different initial messages of Pregel API in GraphX for different types of vertices.
%
This has made the design of our pattern finding algorithm more challenging as we need to think of an initial message that would not mess up the attribute of any vertex.
%
The flexbility of initial messages may ease the implementation in some cases.
%
Secondly, since GraphX uses Pregel API, which is a synchronous model, the framework may not be the best fit for some algorithms, especially in machine learning and/or data mining.
%
In a synchronous model like Pregel, every vertex executes the same iteration.
%
That is, in the case that some of the vertices need more time to finish their vertex programs than the others, the entire computation is blocked.
%
An asynchronous model can make the computation a lot fast as the most recent information is being used, as opposed to previous iteration.
%

\section{Experimental Evaluation}
%
Although the focus of this project is to learn a new graph processing framework, we still describe a little bit how we are going to evaluate our proposed solution.
%
\subsection{Dataset} \label{dataset}
%
We are planning to use the DBLP Computer Science Bibliography dataset, which can be downloaded \href{http://www.rdfhdt.org/datasets/}{here}.
%
The dataset is an RDF graph with 55 million triples.
%
Since our project includes a keyword filtering feature, we are also planning to crawl the paper contents for text processing and analysis.
%
However, the feasibility of crawling papers is to be determined.
%
In the case that crawling papers is not feasible, for the purpose of the project, we will manually download and transform some paper contents to a format that can be easily processed.
%
For the feature that is based on the user reading histories, we plan to use (1) our own reading histories, and (2) synthetic histories.
%
\subsection{Methodology}
%
The evaluation methodologies for both keyword-filtering and reading-hisotry recommendation features are similar.
%
Since evaluating the effectiveness of the recommendations by automatic testing is challenging, we decide to evaluate our results manually.
%
We are going to perform our proposed algorithms on the DBLP dataset described in Section \ref{dataset}.
%
For keyword-filtering recommendation, a keyword from a list of randomly generated keywords is going to be an input.
%
For reading-history recommendation, a reading list is going to be an input.
%
Then, we manually evaluate how relevant the results are to the input keyword or reading history, respectively.
%

\section{Conclusion}

This is conclusion.


\bibliographystyle{ACM-Reference-Format}

\bibliography{bibliography}

\end{document}
