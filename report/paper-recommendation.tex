\documentclass[sigconf]{acmart}

\usepackage{booktabs} % For formal tables
\usepackage{amsmath}
\usepackage{graphicx}
\usepackage[]{algorithm2e}
\graphicspath{ {./} }
\usepackage{listings}

\lstset{
  columns=fullflexible,
  frame=single,
  breaklines=true,
  postbreak=\mbox{\textcolor{red}{$\hookrightarrow$}\space},
}

% Copyright
\setcopyright{none}
%\setcopyright{acmcopyright}
%\setcopyright{acmlicensed}
%\setcopyright{rightsretained}
%\setcopyright{usgov}
%\setcopyright{usgovmixed}
%\setcopyright{cagov}
%\setcopyright{cagovmixed}

\settopmatter{printacmref=false, printccs=false, printfolios=false}

\newcommand\blfootnote[1]{%
  \begingroup
  \renewcommand\thefootnote{}\footnote{#1}%
  \addtocounter{footnote}{-1}%
  \endgroup
}

\begin{document}
\title{Paper Recommendation using GraphX} 

\author{Jeremy Chen}
\affiliation{%
  \institution{University of Waterloo}
}
\email{jeremy.chen@uwaterloo.ca}

\author{Junyi Zhang}
\affiliation{%
  \institution{University of Waterloo}
}
\email{j823zhan@uwaterloo.ca}

\author{Yuqing Xie}
\affiliation{%
  \institution{University of Waterloo} }
\email{yuqing.xie@uwaterloo.ca}

% removes footnote with conference information in first column
\renewcommand\footnotetextcopyrightpermission[1]{}
% removes running headers
\pagestyle{plain}


\begin{abstract} \blfootnote{Project source code can be downloaded \href{https://github.com/amyxie361/CS651\_Paper\_recommendation}{\underline{here}}}
Data in graph structure has become popular nowadays.
%
Graph processing framework, therefore, becomes an essential tool to perform scalable computation.
%
GraphX is one of the top choices because of its easiness, scalability, and performance.
%
In this project, we explore GraphX in a deeper way and gain some insights.
%
We achieve this by using GraphX to build an academic paper recommendation system.
%
Without using the trivial way, we implement PageRank, keyword filtering, and pattern finding algorithms in GraphX.
%
We apply them on a citation network to give recommendations of papers, with taking users' interest into account.
%
We have two metrics to evaluate our recommendations: inversion count and ranking distance, and both have given us positive feedback.
%
Through doing this project, we have not only studied GraphX in a practical way but performed data analytics on a citation network.
%
\end{abstract}

\maketitle 

\section{Introduction}
%
With the rapid publication of scientific literature, conducting a comprehensive literature review has become more challenging.
%
Keeping up with current development of a certain area also requires huge effort.
%
With limited time and energy, researchers would like to invest their time on the papers that can give them the most inspiration.
%
However, without any assistance on filtering papers, finding representable papers in a particular field seems impossible.

%
This is where citation network can be useful.
%
A citation network is a network that tells the relationship between papers in terms of references/citations.
%
Since a citation network is in graph structure, we can leverage existing graph data processing framework to help us analyze the importance of each paper.
%
This motivates us to learn a graph processing framework and build a system to recommend must-read papers.
%
In this project, we explore a new framework \textbf{GraphX}, which makes graph computation easy and scalable.
%
The objective of this project is to learn a new framework and implement network algorithms to accomplish \textbf{paper recommendation}.
%
The main goal of the system is to recommend users related papers based on input keywords, citation network and reading history.
%

The rest of the paper is strctured as follows. 
%
Section \ref{sec:related-work} reviews related work on recommendation. 
%
We define the problem that we need to tackle in Section \ref{sec:problem}.
%
Section \ref{sec:framework} discusses about the framework we use (i.e. GraphX), and we describe the algorithms we use in GraphX in Section \ref{sec:algo}.
%
In Section \ref{sec:insights}, we analyze some of the advantages and disadvantages of GraphX and discuss what we have learned while exploring the framework.
%
We present our experimental results in Section \ref{sec:exp}.
%
Section \ref{sec:conclusion} states some future work of this project and concludes the paper.

\section{Related Work}

There are number of researches on recommendation system, and the methods generally could be divided into machine learning and graph. Our approach is based on graph, one relevant work would be 
a service that provides real-time recommendations to tens of millions of mobile users in Twitter by implementing real-time motif detection on large dynamic graphs ~\cite{Gupta:2014}. In addition, similarity measures ~\cite{Goel2013} have been extensively studied in the area of information retrieval and networks, i.e. user similarity computation and the effects of user similarity in social media. Similarities between two entities are used for community detection, similarity search and recommendations, and most of them proposed for networks are based on graph structure, i.e. SimRank ~\cite{Jeh:2002}, Penetrating Rank ~\cite{Zhao:2009}. 

The papers mentioned above and our project are all based on large-scale graph processing. One suitable system for this kind of problem is Pregel ~\cite{Malewicz:2010}, where programs are expressed as a sequence of iterations. In each iteration, a vertex could receive the message sent from the previous iteration, then it sends messages to other vertices, updates its own state and outgoing edges. Since large-scale graphs that with millions of vertices and billions of edges are difficult to analyze, researchers have usually turned to distributed solutions, i.e. MapReduce ~\cite{Lin:2010}, or reduce the size of the graph by partitioning \ref{Karypis:1998}.

Furthermore, there is a pretty related work to our project ~\cite{DBLP}, which is about content-based citation recommendation. However, the method used in this paper is neural model instead of graph. They first embed all available documents into a vector space, and encoded text content of each document. Then, chose the nearest neighbours of the query document as candidates and reranked the candidates by another model trained, where this model takes a pair of documents as input and estimates the probability that document2 should be cited in document1.

The evaluation methods ~\cite{Beel:2013} for recommendation system include online evaluations and offline evaluations. Online evaluation means that recommendations are shown to users as they use the real-world system, while offline evaluation ~\cite{Ricci:2010} includes precision, recall and F-measure. A good recommender system contributes to three features, recommendation accuracy, user satisfaction, and provider satisfaction, and leads to the question how these three features are to be quantified and compared.
\section{Problem Description}

The goal of the project is to recommend papers to researchers according to their interest. There are two main features to represent users' interest. The first will be keywords. The second will be researcher's reading history.  
%We will also consider other interesting patterns according to experiments.

%We will solve the problem of paper recommendation based on keywords, citation network and user reading history. 
We will have an offline dataset consisting of paper citation relation and papers' content.

Next, we give formal notations to describe our problem. We separate the problem into two part: given the interested keywords then the system output a list of recommending papers, or given a list of read papers in user's history then the system output a list of recommended papers. 

\subsection{First Part: Recommend According to Keywords}

For the first part, the input will be a list of $n$ strings $K = [k_1, k_2, ..., k_n]$, where each string is an interested field, e.g. ["machine learning", "computer vision"]. The desired paper list will be $R = [r_1, r_2, ..., r_m]$, in the form of indexs of $m$ papers. Our system will recommend papers in an ordered way, which means $r_i$ is a better paper than $r_{i+1}, \forall i$.

\subsection{Second Part: Recommend According to Reading History}

For the second part, the input will be a list of papers $P = [p_1, p_2, ..., p_n]$, where each paper is represented as a unique id, e.g. ["journals/cacm/Szalay08"]. This represents user's reading history ignoring the time sequence. The desired paper list will be $R =  [r_1, r_2, ..., r_m]$, in the form of indexs of $m$ papers. Our system will also recommend papers in an ordered way, which means $r_i$ is a better paper than $r_{i+1}, \forall i$.

\subsection{Graph representation}

First of all, we represent the citation relations among papers to be a directed graph $G=(V, E)$. $V$ is the vertex set, each vertex represents a paper, we use the index to represent each paper. We also stored the detailed information of papers in a separated map from index to information. $E$ is the directed edge set. $\exists e=(v -> u) \in E$ means the paper $v$ cites the paper $u$.

\subsubsection{Subgraph}
We also need the concept of subgraph $G_s = (V_s, E_s)$. In the subgraph, we ensure that $V_s \subseteq V$, meaning we keep a subset of the original graph and get rid of the other vertexes. Then we compute the sub-edge set according to the sub-vertexes set: $E_s = \{e=(u -> v) | u \in V_s and v \in V_s\ and e \in E\}$. In this work, we do not need other subgraphs, which means a subgraph is determined by the original graph $G$ and the sub-vertexes set $V_s$.

\subsubsection{Reverse Graph}
Since we are working on a directed graph, we can define the concept of a reversed graph. $G^*=(V^*, E^*)$ is a reversed graph of $G=(V, E)\iff $
\begin{equation}
 V^* = V \text{, and } E^* = \{(u -> v)| (v->u) \in E\}.
\end{equation}

\subsubsection{Patterns}

According to previous network studies {TODO, related work about patterns}, several common patterns are extremely useful for recommendation on networks. We address two main patterns: the diamond pattern and the triangle pattern. 

\paragraph{Diamond pattern}

\begin{figure}[t]
	\centering
	\includegraphics[width=0.7\linewidth]{diamond.pdf}
	\caption{Example of a diamond pattern.}
	\label{fig:diamond}
\end{figure}

As shown in Figure \ref{fig:diamond}, a diamond pattern is that if a user has read paper $A$, and paper $A$ cites both paper $B$ and $C$, paper $B$ and $C$ both cite paper $D$. Then, we will recommend paper $D$ to the user if the paper has not been read. This is easy to understand because paper $D$ may be an important work in the reader's interested field.

\paragraph{Triangle pattern}

\begin{figure}[t]
	\centering
	\includegraphics[width=0.7\linewidth]{triangle}
	\caption{Example of a triangle pattern.}
	\label{fig:triangle}
\end{figure}

Triangle is also easy to understand. As shown in Figure \ref{fig:triangle}, a triangle pattern is that if a user has read paper $A$, and paper $A$ cites both paper $B$ and $C$, paper $B$ cites paper $C$. Then, we will recommend paper $C$ to the user if the paper has not been read. 


\section{GraphX}
%
In this section, we discuss about the big data processing framework for graphs, GraphX, that we use to tackle our problem.
%
We will highlight some of the features in GraphX and go through some of the essential details of Pregel API.
%
\subsection{Highlights}
%
GraphX is a component built on top of Spark therefore has adopted many benefits and features from Spark.
%
In Spark, data to be processed is represented as Resilient Distributed Dataset (RDD), which is immutable and distributed.
%
In GraphX, graph data is abstracted as a directed graph allowing properties associated with vertices and edges.
%
GraphX extends Spark RDD to represent a graph.
%
In particular, a graph is represented as VertexRDD and EdgeRDD.
%
Since VertexRDD and EdgeRDD are they are simply extensions of RDD, operations that can be applied on RDD can also be applied on them.
%
\subsection{Pregel API}
%
In terms of the richness of operators, GraphX provides a set of fundamental operators such as "subgraph," "reverse edges," etc. as well as a few common graph algorithms such as PageRank, ConnectedComponents, etc..
%
One feature that we use heavily in GraphX is the Pregel API.
%
Pregel API uses the concept of "thinking like a vertex."
%
In other words, the API provides a vertex-centric way of processing graphs.
%
Users are required to specify the following functions for a Pregel API call.
%
\begin{itemize}
%
\item \textbf{Vertex Program} (denoted as $vprog$): 
%
This function runs on every vertex of the graph.
%
In this function, each vertex $V$ has the access of the aggreated messsage (see below) delivered to $V$ as well as the attribute of $V$.
%
The function returns the new state (i.e. attribute) of the vertex.
%
\item \textbf{Send Message Program} (denoted as $sendMsg$):
%
This function runs on the edges in the EdgeRDD. 
%
The input of this function is an edge triplet, which is a view logically joining the vertex and edge properties
%
The function returns a message to be sent to the neighbors of each vertex.
%
\item \textbf{Merge Message Program} (denoted as $mergeMsg$):
%
This function combines two messages that are sent to the same vertex in an iteration into one.
%
The function returns the combined message.
%
Because of this aggregation, when a vertex executes the vertex program in the next iteration, only the aggreated message is passed in.
%
\end{itemize}
%
In addition, users need to specify the following parameters.
%
\begin{itemize}
%
\item \textbf{Initial Message} (denoted as $initialMsg$):
%
This is the message passed to every vertex in the first iteration.
%
The purpose of this message is to make sure each vertex executes the vertex program at least once.
%
\item \textbf{Max Iteration} (denoted as $maxIterations$):
%
This specifies the maximum number of iterations that can be run for a Pregel API call.
%
\item \textbf{Edge Direction} (denoted as $activeDirection$):
%
This specifies the direction of the edges on which the $sendMsg$ function will run.
%
\end{itemize}
%
As a summary of how Pregel works, there are two major parts of the computation in each iteration.
%
\begin{enumerate}
%
\item \textbf{Execution of Vertex Program}: 
%
On each vertex, the vertex program would be run sequentially.
%
Note that only those vertices that have incoming messages from previous iteration will be run.
%
\item \textbf{Message Preparation}:
%
A message that will be sent would be constructed and delivered at the beginning of next iteration.
%
\end{enumerate}
%
We express the execution of a Pregel API call at a high level in Algorithm \ref{alg:pregel}.
%
\begin{algorithm}
  \SetKwInOut{Input}{Input}
  \SetKwInOut{Output}{Output}
  \SetKwInOut{Init}{Init}
  \Input{Initial Graph: $G = (V, E)$}
  \Output{Updated Graph: $G = (V, E)$}
  \SetAlgoLined
  \SetKwProg{Fn}{Function}{:}{end}
  \Fn{Pregel(initialMsg, maxIterations, activeDirection, vprog, sendMsg, mergeMsg)}{
     numIterations := 0 \;
     initialMsg is sent to each vertex \;
     \While{\# of vertices that will receive msg > 0 OR numIterations < maxIterations}{
       vprog(...) // run in parallel on each vertex \;
       sendMsg(...) // run in parallel on each edge satisfying activeDirection \;
       mergeMsg(...) // run in parallel on each set of messages sent to the same vertex \;
     }
  }
  \caption{Pregel API}
  \label{alg:pregel}
\end{algorithm}
%


\section{Algorithms}

For the separated two parts of programs, we develop our algorithm differently. 

\subsection{Keywords recommendation algorithm}

Before implement recommendation algorithm, we first implemented the TF/IDF algorithm to preprocessing papers' abstracts into their keywords. Then we implemented the classic PageRank algorithm using GraphX for recommendation score computation.

\subsubsection{Preprocessing}

For computation effectiveness, we only keep the abstract of each paper. We first tokenize the abstract, then implemented the TF/IDF algorithm. Now we acquired the TF/IDF score for each token in each paper. Then we filter $k$ tokens with the highest score in each paper to be the paper's keywords. Then we construct an inverse map from keywords to paper index, meaning the token is a keyword for those papers. 
%We will also take the original author-generated keywords into consideration.

\subsubsection{PageRank for recommendation}

After we get the keywords to papers map, we implement the classic page rank algorithm to compute the score. 

First, according to the keywords from the input, we filter out the subgraphs. Each paper in this subgraph have the target tokens as their keywords. 

Next, we will first use GraphX to construct the citation network.  After we constructed the graph, we will implement the classic PageRank algorithm to establish a paper ranking. You can refer to \ref{alg:pagerank} to see detailed GraphX inplementation

\begin{algorithm}
  \SetKwInOut{Input}{Input}
  \SetKwInOut{Output}{Output}
  \SetKw{KwForEach}{foreach}
  \SetKw{KwIn}{in}
  \Input{Keywords list: $K = [k_1, k_2, ..., k_n]$, $G=(V, E)$, Term2PapersIndex: $T$ -> \{paper IDs with $T$ as a keyword\}}
  \Output{Filtered graph: $G=(V, E)$}
  \SetKwProg{Fn}{Function}{:}{end}
  \Fn{FilterGraph(Keywords)}{
    intersectedPaperIds := Intersection( \\
      Term2PapersIndex[$k_1$], \\
      Term2PapersIndex[$k_2$], \\
      ..., \\
      Term2PapersIndex[$k_n$] \\
    ) \\
    return subgraph of G containing IDs in intersectedPaperIds and associated edges
  }
  \caption{Keyword Filtering Algorithm}
  \label{alg:filter}
\end{algorithm}

\begin{algorithm}
 \SetKwInOut{Input}{Input}
 \SetKwInOut{Output}{Output}
 \SetKwInOut{Init}{Init}
 \SetKwProg{Fn}{Function}{:}{end}
 \Input{Keywords list: $K = [k_1, k_2, ..., k_n]$, $G=(V, E)$}
 \Output{Recommended paper list: $R = [r_1, r_2, ..., r_m]$}
 %initial
 \Fn{InitialGraph}{
 	JoinVertices\\
	map(edge.Attr =>$\frac{1}{edge.srcAttr}$)\\
	map(Vertice => 1.0)
 }
Init.Message $\gets$ 0.0\\
G $\gets$ \textbf{InitialGraph}.Filter(K)\\
set resetProb\\
set numIteration\\
\Fn{\textbf{vProg}(Vertex, sumMsg)}{
 	\Return{resetProb + (1.0 - resetProb) * msgSum}
 }
\Fn{\textbf{sendMsg}(Edge)}{
 	\Return{edge.srcAttr * edge.Attr}
 }
 \Fn{\textbf{mergeMsg}(aMsg, bMsg)}{
 	\Return{aMsg + bMsg}
 }
 G $\gets$ G.Pregel(initialMsg, maxIteration=numIteration, activateDirection=out, vProg, sendMsg, mergeMsg)\\
R $\gets$ G.Vertex.SortedBy(Vertex.Attr)\\
\Return{R}\\
 \caption{Paper Rank Algorithm}
 \label{alg:pagerank}
\end{algorithm}

Then we give recommendation paper list according to the ranking score, from the highest to the lowest.

\subsubsection{Pattern mining for recommendation}

To take advantage of the reading history for recommendation, we use pattern mining algorithms. 

We will find common patterns such as diamond pattern and triangle pattern to do recommendation. To find the interested classic papers, we find patterns on the original graph. To find the interested recent paper, we find the patterns on the reversed graph.

To simplify the description, we first consider the original graph. The recent paper version will be the same after we turn the graph into a reversed version.

Because the restrictions from GraphX, we design the algorithm to be a two-iteration algorithm. And the operations of each iteration will be different. In the first iteration, we filter out the papers cited by the papers in the reading history. We call them referred papers. In the second iteration, we mine the patterns out. You can refer to the detailed algorithm in Algorithm \ref{alg:pattern}

\begin{algorithm}
 \SetKwInOut{Input}{Input}
 \SetKwInOut{Output}{Output}
 \SetKwInOut{Init}{Init}
 \SetKwProg{Fn}{Function}{:}{end}
 \Input{Paper list: $P = [p_1, p_2, ..., p_n]$, $G=(V, E)$}
 \Output{Recommended paper list: $R = [r_1, r_2, ..., r_m]$}
 %initial
 \Fn{\textbf{InitialGraph}(P)}{
 	\eIf{Vertex.Id $\in$ P}{Vertex.Attr $\gets$ 1.0}
	{Vertex.Attr $\gets$ 0.0}
 }
Init.Message $\gets$ 0.0\\
G $\gets$ InitialGraph(P)\\
\Fn{\textbf{vProg1}(Vertex, sumMsg)}{
 	\Return{Vertex.Attr + sumMsg}
 }
\Fn{\textbf{sendMsg1}(Edge)}{
	\eIf{src.Attr > 0.0}{\Return{2.0}}{
 	\Return{0.0}}
 }
 \Fn{\textbf{mergeMsg1}(aMsg, bMsg)}{
 	\Return{$\max$(aMsg, bMsg)}
 }
 G $\gets$ G.Pregel(initialMsg, maxIteration=1, activateDirection=out, vProg1, sendMsg1, mergeMsg1)\\
 \Fn{\textbf{mapVertex}(Vertex)}{
 	\eIf{Vertex.Attr > 1.0}{Vertex.Attr$\gets$(1.0, 0.0)}{Vertex.Attr$\gets$(0.0, 0.0)}
 }
G $\gets$ G.mapVertex(mapVertex,Vertex)\\
\Fn{\textbf{vProg2}(Vertex, sumMsg)}{
 	\Return{(Vertex.Attr.\_1, sumMsg)}
 }
\Fn{\textbf{sendMsg2}(Edge)}{
	\eIf{src.Attr.\_1 > 0.0}{\Return{1.0}}{
 	\Return{0.0}}
 }
 \Fn{\textbf{mergeMsg2}(aMsg, bMsg)}{
 	\Return{aMsg + bMsg}
 }
 G $\gets$ G.Pregel(initialMsg, maxIteration=1, activateDirection=out, vProg2, sendMsg2, mergeMsg2)\\
R $\gets$ G.Vertiex.filter(Vertex.Attr.\_1 + Vertex.Attr.\_2 $\geq$ 2.0)\\
\Return{R}\\
 \caption{Pattern Mining Algorithm}
 \label{alg:pattern}
\end{algorithm}






\section{Insights} \label{sec:insights}
%
Since the main component of this project consists of learning the framework GraphX, we list some insights and what we have learned whiling exploring it.
%
\subsection{Advantages of GraphX}
%
\begin{figure}[t]
	\centering
	\includegraphics[width=0.7\linewidth]{edge_cut_vs_vertex_cut.png}
	\caption{Edge-based partitioning VS Vertex-based partitioning.}
	\label{fig:edgecut}
\end{figure}

Since GraphX is built on top of Spark, it adopts a lot of advantages from Spark.
%
One of the biggest advantages of GraphX is the easiness of implementation.
%
As the purpose of this project is to learn the framework, we choose not to use the PageRank function provided by GraphX.
%
Instead, we implement the PageRank algorithm by ourselves using Pregel API in GraphX.
%
However, the setup in order to initiate the execution of PageRank using Pregel only takes approximately 20 lines of code to complete.
%
Then, GraphX handles everything else for us, including fault-tolerance, concurrency control, etc..
%

There is one significant difference between the implementation of Pregel API in GraphX and standard Pregel implementation.
%
The Pregel API in GraphX only allows sending messages to the neighbors, whereas standard Pregel implementation can send messages to any vertex in the graph.
%
Although this sounds like a disadvantage, this "limitation" helps GraphX improve the efficiency.
%
Since in GraphX a graph is an extension of RDD, a graph dataset would be distributed across multiple machines.
%
Intuitively, we could partition a graph based on the vertices(Figure \ref{fig:edgecut}(a))\cite{Gonzalez:2014:GGP:2685048.2685096}, for example, by hashing vertex IDs. 
%
However, this would cause large communication overhead as a lot of messages need to be sent to across machines, which becomes a bottleneck of the computation.
%
Instead, GraphX handles graph partitioning in a edge-based manner (Figure \ref{fig:edgecut}(b)).
%
Edges are partitioned onto different machines and vertices are allowed to span on multiple machines.
%
This way, with the "limitation" that messages can only be sent to direct neighbors, we can partition the graph in a way that the communication overhead is very small.
%

\begin{table*}[ht]
	\centering
	\begin{tabular}{lccp{6.5cm}c}
		\toprule
		\textbf{Keywords}	& \textbf{Inverse count} 	& \textbf{Rank distance} 	&\textbf{Top one title} &\textbf{Top one citation}\\ \midrule
		Machine Learning	& 8					& 1.75			&Support-Vector Networks &33820\\
		Cryptography Security& 5					& 1.125			&State of the Art in Ultra-Low Power Public Key Cryptography for Wireless Sensor Networks &268 \\
		Distributed System	& 8					& 1.75			&Distributed Systems: Principles and Paradigms & 4038\\
		Deep Learning		& 11					& 2.25			&Deep learning via semi-supervised embedding &610\\
		Computer Vision 	& 8					& 1.5				&OpenVIDIA: parallel GPU computer vision & 281\\
		\bottomrule
	\end{tabular}
	\vspace{3mm}
	\caption{Results for selected keywords recommendation. }
	\label{res:keywordall}
\end{table*}

\begin{table*}[ht]
	\centering
	\begin{tabular}{lp{12cm}cc}
		\textbf{Keywords}	& Machine Learning \\ \hline
		\toprule
		\textbf{Rank}		& \textbf{Paper Title} 		& \textbf{Citation} 	&\textbf{Publish year}\\ \midrule
		1				&Support-Vector Networks 	&33820 &1995\\
		2				&Learning with Kernels: Support Vector Machines, Regularization, Optimization, and Beyond &10802&2001\\
		3				&Machine learning in automated text categorization &9095&2002\\
		4				&Application of argument based machine learning to law &3&2005\\
		5				&Data Mining: Practical Machine Learning Tools and Techniques&35014&2004\\
		6				&Very Simple Classification Rules Perform Well on Most Commonly Used Datasets&2060&1993\\
		7				&Investigating statistical machine learning as a tool for software development &59&2008\\
		8				&Machine Learning for User Modeling & 474&2001\\
		\bottomrule
	\end{tabular}
	\vspace{3mm}
	\caption{An example for keywords recommendation. }
	\label{res:keywordexp}
\end{table*}

\subsection{Disadvantages of GraphX}
%
While implementing algorithms in GraphX, we have also found some disadvantages of the framework.
%
First, we are not allowed to have different initial messages of Pregel API in GraphX for different types of vertices.
%
This has made the design of our pattern finding algorithm more challenging as we need to think of an initial message that would not mess up the attribute of any vertex.
%
The flexbility of initial messages may ease the implementation in some cases.
%
Secondly, since GraphX uses Pregel API, which is a synchronous model, the framework may not be the best fit for some algorithms, especially in machine learning and/or data mining.
%
In a synchronous model like Pregel, every vertex executes the same iteration.
%
That is, in the case that some of the vertices need more time to finish their vertex programs than the others, the entire computation is blocked.
%
An asynchronous model can make the computation a lot fast as the most recent information is being used, as opposed to previous iteration.
%

\section{Experimental Evaluation}
%
Although the focus of this project is to learn a new graph processing framework, we still describe a little bit how we are going to evaluate our proposed solution.
%
\subsection{Dataset} \label{dataset}
%
We are planning to use the DBLP Computer Science Bibliography dataset, which can be downloaded \href{https://dblp.uni-trier.de/xml/}{\underline{here}}.
%
The raw dataset is in XML format, and we will convert it into a graph structure.
%
Since our project includes a keyword filtering feature, we are also planning to crawl the paper contents for text processing and analysis.
%
For each paper in the DBLP data, a link to the entry page of the paper has been provided.
%
We can crawl the contents of the entry page and obtain the abstract as an offline dataset for text processing.
%
For the feature that is based on the user reading histories, we plan to use (1) our own reading histories, and (2) synthetic histories.
%
\subsection{Evaluation Methodology}
%
The evaluation methodologies for both keyword-filtering and reading-hisotry recommendation features are similar.
%
Since evaluating the effectiveness of the recommendations by automatic testing is challenging, one way to evaluate our results is manual evaluation. In this work, we face the classic cold start problem in recommendation system: there is no good data for interested papers according to keywords or reading history. We will use both manually evaluation and citation rank evaluation for results measuring.
%
We are going to perform our proposed algorithms on the DBLP dataset described in Section \ref{dataset}.
%
For keyword-filtering recommendation, a keyword from a list of randomly generated keywords is going to be an input.
%
For reading-history recommendation, a reading list is going to be an input.
%
Then, we manually evaluate how relevant the results are to the input keyword or reading history, respectively.
%
The other way to evaluate our results is based on the number of citations of a paper. The recommendations are effective if they are among the most-cited papers.
%
Given a recommendation ranking list $r_1, r_2, ..., r_m$, we crawl their citations $c_1, c_2, ..., c_m$. Then we re-rank the papers according to the citation, from the most to the least. The ranked papers will be $r_{i_1}, r_{i_2},..., r_{i_m}$. Then we count the number of inversion in the new index ranking. We also measure the rank distance: $D(i_1, ..., i_m) = \frac{1}{m} \sum_{j=1}^m |i_j - j|$.
%


\subsection{Experimental Results}
%
Because of the cold start problem, we display out results through examples. For the keywords recommendation, we select the top 8 papers to be our recommendation. For the reading history recommendation, we randomly select paper in related field and feed these index to be the input. We select the top 5 papers from pattern to be our recommendation.

\subsubsection{Keywords recommendation results}

\begin{table}
	\centering
	\begin{tabular}{lccp{6.5cm}c}
		\toprule
		\textbf{Keywords}	& \textbf{Inverse count} 	& \textbf{rank score} 	&\textbf{Top one title} &\textbf{Top one citation}\\ \midrule
		Machine Learning	& 8					& 1.75			&Support-Vector Networks &33820\\
		Cryptography Security& 5					& 1.125			&State of the Art in Ultra-Low Power Public Key Cryptography for Wireless Sensor Networks &268 \\
		Distributed System	& 8					& 1.75			&Distributed Systems: Principles and Paradigms & 4038\\
		Deep Learning		& 11					& 2.25			&Deep learning via semi-supervised embedding &610\\
		Computer Vision 	& 8					& 1.5				&OpenVIDIA: parallel GPU computer vision & 281\\
		\bottomrule
	\end{tabular}
	\vspace{3mm}
	\caption{Results for selected keywords recommendation. }
	\label{res:keywordall}
\end{table}


\begin{table}
	\centering
	\begin{tabular}{lp{12cm}c}
		\textbf{Keywords}	& Machine Learning \\ \hline
		\toprule
		\textbf{Rank}		& \textbf{Paper Title} 		& \textbf{Citation} 	\\ \midrule
		1				&Support-Vector Networks 	&33820\\
		2				&Learning with Kernels: Support Vector Machines, Regularization, Optimization, and Beyond &10802\\
		3				&Machine learning in automated text categorization &9095\\
		4				&Application of argument based machine learning to law &3\\
		5				&Data Mining: Practical Machine Learning Tools and Techniques&35014\\
		6				&Very Simple Classification Rules Perform Well on Most Commonly Used Datasets&2060\\
		7				&Investigating statistical machine learning as a tool for software development &59\\
		8				&Machine Learning for User Modeling & 474\\
		\bottomrule
	\end{tabular}
	\vspace{3mm}
	\caption{An example for keywords recommendation. }
	\label{res:keywordexp}
\end{table}

\subsubsection{Reading History recommendation results}

\begin{table}
	\centering
	\begin{tabular}{lccc}
		\toprule
		\textbf{Exp ID} 	& \textbf{Inverse count} 	& \textbf{rank score} &\textbf{Top one citation}\\ \midrule
		1	& 2	& 0.8		&75881\\
		2	& 4	& 1.6		& 96 \\
		3	& 1	& 0.4		&20024 \\
		4	& 3	& 1.6		&2796\\
		5	& 3	& 1.2		&10141\\
		\bottomrule
	\end{tabular}
	\vspace{3mm}
	\caption{Results for reading history recommendation. }
	\label{res:patternall}
\end{table}

\begin{table}
	\centering
	\begin{tabular}{p{6cm}p{6cm}c}
		\toprule
		\textbf{Input paper}		& \textbf{Recommend Paper Title} 		& \textbf{Recommend Paper Citation} 	\\ \midrule
		State of the Art in Ultra-Low Power Public Key Cryptography for Wireless Sensor Networks &A method for obtaining digital signatures and public-key cryptosystems	&20024\\
		Cryptography and Network Security: Principles and Practice				&Handbook of Applied Cryptography &18008\\
		Computational soundness for standard assumptions of formal cryptography				&Securing ad hoc networks &3625\\
		Minimalist cryptography for low-cost RFID tags (extended abstract)				&The Resurrecting Duckling: Security Issues for Ad-hoc Wireless Networks &1639\\
		Securing Mobile Ad Hoc Networks with Certificateless Public Keys				&Mitigating routing misbehavior in mobile ad hoc networks&4594\\
		BeeHiveGuard: a step towards secure nature inspired routing algorithms\\
		A proposed curriculum of cryptography courses\\
		Integration of Quantum Cryptography in 802.11 Networks\\
		\bottomrule
	\end{tabular}
	\vspace{3mm}
	\caption{An example for reading history recommendation. }
	\label{res:patternexp}
\end{table}

\subsection{Experimental Analysis}

\begin{table}
	\centering
	\begin{tabular}{llll}
		\toprule
		\textbf{Token}		& \textbf{\# Appearance as keywords} \textbf{Token}		& \textbf{\# Appearance as keywords} \\ \midrule
		web	&23293 &service&17037\\
		image &23082&d&15955\\
		network &22428&algorithm&15906\\
		data &22134&video&15294\\
		software &21866 &services&15019\\
		learning &21299&search&14931\\
		mobile &18176&you&14912\\
		fuzzy &17559&students&14236\\
		control &17389&sensor&13855\\
		security & 17060&graph&13686\\
		\bottomrule
	\end{tabular}
	\vspace{3mm}
	\caption{Top 20 keywords appears in papers. }
	\label{res:tfidf}
\end{table}


\section{Conclusion} \label{sec:conclusion}

This is conclusion.

Future work:
According to the previous analysis, we could improve the system in several ways. First we can include external TF/IDF score or compute the score according to the whole paper to eliminate problem from the language unicity. In this way, we can acquire better keywords. Second, we can try other valid citation network dataset for more updated data. Also, we can explore other pattern mining method and try different recommendation algorithms. Another very important work should be done is that we have to solve the cold start problem and develop auto evaluation process for unlabeled recommendation problem. 

\bibliographystyle{ACM-Reference-Format}

\bibliography{bibliography}

\end{document}
