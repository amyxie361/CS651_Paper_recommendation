\section{Experimental Evaluation} \label{sec:exp}
%
Although the focus of this project is to learn a new graph processing framework, we still describe a little bit how we are going to evaluate our proposed solution.
%
\subsection{Dataset} \label{dataset}
%
We use the ACM Citation Network as our graph dataset, which can be downloaded \href{https://aminer.org/citation}{\underline{here}}.
%
The raw dataset is in text file, and it includes the information we need such as references of papers, abstracts, etc..
%
As the raw dataset uses strings as paper unique IDs, we first transform the dataset to have integer IDs starting from 1 to ease our processing.
%
We then extract the graph topology (for PageRank and pattern finding algorithms) and the paper abstracts (for TF-IDF) to construct the datasets in our desired format.
%
\begin{table*}[ht]
	\centering
	\begin{tabular}{p{7cm}p{7cm}p{3cm}}
		\toprule
		\textbf{Input paper}		& \textbf{Recommend Paper Title} 		& \textbf{Recommend Paper Citation} 	\\ \midrule
		State of the Art in Ultra-Low Power Public Key Cryptography for Wireless Sensor Networks &A method for obtaining digital signatures and public-key cryptosystems	&20024\\
		Cryptography and Network Security: Principles and Practice				&Handbook of Applied Cryptography &18008\\
		Computational soundness for standard assumptions of formal cryptography				&Securing ad hoc networks &3625\\
		Minimalist cryptography for low-cost RFID tags (extended abstract)				&The Resurrecting Duckling: Security Issues for Ad-hoc Wireless Networks &1639\\
		Securing Mobile Ad Hoc Networks with Certificateless Public Keys				&Mitigating routing misbehavior in mobile ad hoc networks&4594\\
		BeeHiveGuard: a step towards secure nature inspired routing algorithms\\
		A proposed curriculum of cryptography courses\\
		Integration of Quantum Cryptography in 802.11 Networks\\
		\bottomrule
	\end{tabular}
	\vspace{3mm}
	\caption{An example for reading history recommendation. }
	\label{res:patternexp}
\end{table*}
%
\subsection{Evaluation Methodology}
%
The evaluation methodologies for both keyword-filtering and reading-hisotry recommendation features are similar.
%
Since evaluating the effectiveness of the recommendations by automatic testing is challenging, one way to evaluate our results is manual evaluation. In this work, we face the classic cold start problem in recommendation system: there is no good data for interested papers according to keywords or reading history. We will use both manually evaluation and citation rank evaluation for results measuring.
%
We are going to perform our proposed algorithms on the DBLP dataset described in Section \ref{dataset}.
%
For keyword-filtering recommendation, a keyword from a list of randomly generated keywords is going to be an input.
%
For reading-history recommendation, a reading list is going to be an input.
%
Then, we manually evaluate how relevant the results are to the input keyword or reading history, respectively.
%
The other way to evaluate our results is based on the number of citations of a paper. The recommendations are effective if they are among the most-cited papers.
%
Given a recommendation ranking list $r_1, r_2, ..., r_m$, we crawl their citations $c_1, c_2, ..., c_m$. Then we re-rank the papers according to the citation, from the most to the least. The ranked papers will be $r_{i_1}, r_{i_2},..., r_{i_m}$. Then we count the number of inversion in the new index ranking. We also measure the rank distance: $D(i_1, ..., i_m) = \frac{1}{m} \sum_{j=1}^m |i_j - j|$.
%

\subsection{Experimental Results}
%
Because of the cold start problem, we display out results through examples. For the keywords recommendation, we select the top 8 papers to be our recommendation. For the reading history recommendation, we randomly select paper in related field and feed these index to be the input. We select the top 5 papers from pattern to be our recommendation.

\subsubsection{Keywords recommendation results}

Table \ref{res:keywordall}, showed the five queries we sent to our system. We display the inverse counts and the rank scores. We also provide the first recommended paper title and it's citation number. The average inverse counts is 8 while the average rank distance is 1.68. The results are quite interesting. For keywords that have relatively long research history, the recommended paper are pretty good. The top one recommendation of machine learning is a work with up to 33,820 citations. As we all know, the support vector related methods was a heat back to the 90's. 

Here we can take a closer look at the machine learning query's return in Table \ref{res:keywordexp}. The selected papers are almost all classic works with high citations. This is a natural result since we propagate the weights to the citation source. Classic papers will finally receive almost all scores from the later papers. We also examined the papers in the middle of the recommendation list. However, these papers still contain some relatively old papers. The main reason may due to the delay of the ACM dataset. We also manually checked several classic but recent papers from our knowledge, but many them either are not in the original dataset or was discarded because of missing of the abstract. 

\subsubsection{Reading History recommendation results}

Table \ref{res:patternexp} shows an example of recommendation query according to reading history. In the first column are the papers in the reading history, while in the second column are the papers we recommended. The input papers are selected from the cryptography security. The average citations of the input papers is 183.9. We applied the original direction pattern mining. As you can see, the recommended papers are all closely related to cryptography and their citations are all above 1,500, which is much more higher than the input papers.

Table \ref{res:patternall} shows statistics on 5 reading history queries. The inverse count and the rank distance are all low. This means the effectiveness of the pattern mining method.

\begin{table}
	\centering
	\begin{tabular}{lccc}
		\toprule
		\textbf{Exp ID} 	& \textbf{Inverse count} 	& \textbf{Rank distance} &\textbf{Top one citation}\\ \midrule
		1	& 2	& 0.8		&75881\\
		2	& 4	& 1.6		& 96 \\
		3	& 1	& 0.4		&20024 \\
		4	& 3	& 1.6		&2796\\
		5	& 3	& 1.2		&10141\\
		\bottomrule
	\end{tabular}
	\vspace{3mm}
	\caption{Results for reading history recommendation. }
	\label{res:patternall}
\end{table}

\subsection{Experimental Analysis}

We can see from the previous results, our system works especially well on mining the classic papers. It can easily find the most popular and ancient papers. This is very important for researchers to find the source and follow the history within a field. Our system may not recommend recent papers. We next analyze the source of these disadvantages.

We first check the keywords selection part. After we extract the top 20 keywords that appears the most in the keywords list, we find an interesting phenomenon. Some common words may appear as keywords frequently, which is not what we expected. You can see from Table \ref{res:tfidf}, the token ``you'' even appear as a keyword in up to 14,912 papers. We then go back to some of these papers. We found that some researchers prefer using ``you'' a lot while the other do not. This leads to the relatively small document frequency of the token while large term frequencies in some papers. This will cause our system not to identify the keywords precisely.

\begin{table}
	\centering
	\begin{tabular}{lp{2.5cm}lp{2.5cm}}
		\toprule
		\textbf{Token}		& \textbf{\# Appearance as keywords} &\textbf{Token}		& \textbf{\# Appearance as keywords} \\ \midrule
		web	&23293 &service&17037\\
		image &23082&d&15955\\
		network &22428&algorithm&15906\\
		data &22134&video&15294\\
		software &21866 &services&15019\\
		learning &21299&search&14931\\
		mobile &18176&you&14912\\
		fuzzy &17559&students&14236\\
		control &17389&sensor&13855\\
		security & 17060&graph&13686\\
		\bottomrule
	\end{tabular}
	\vspace{3mm}
	\caption{Top 20 keywords appears in papers. }
	\label{res:tfidf}
\end{table}

Since the most papers we recommended are all ``ancient'' papers. We dig deep into the dataset. Although the dataset was collect until 2016, the most papers are published before 2010. So the fashion topics such as computer vision and natural language processing will not receive a good performance, especially when people prefer more recent papers. Also, the recommended papers tend to be old due to the processing logic. We also applied the reverse directional reading history recommendation, but the test results are not desirable because of the cold start. That is to say, we do not have an actual reading history which can be applied within this dataset. So the pattern mining becomes difficult due to the poor input selection.

