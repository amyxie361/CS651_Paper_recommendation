\section{Introduction}
%
With the rapid publication of scientific literature, conducting a comprehensive literature review has become more challenging.
%
Keeping up with current development of a certain area also requires huge effort.
%
With limited time and energy, researchers would like to invest their time on the papers that can give them the most inspiration.
%
However, without any assistance on filtering papers, finding representable papers in a particular field seems impossible.

%
This is where citation network can be useful.
%
A citation network is a network that tells the relationship between papers in terms of references/citations.
%
Since a citation network is in graph structure, we can leverage existing graph data processing framework to help us analyze the importance of each paper.
%
This motivates us to learn a graph processing framework and build a system to recommend must-read papers.
%
In this project, we explore a new framework \textbf{GraphX}, which makes graph computation easy and scalable.
%
The objective of this project is to learn a new framework and implement network algorithms to accomplish \textbf{paper recommendation}.
%
The main goal of the system is to recommend users related papers based on input keywords, citation network and reading history.
%

The rest of the paper is strctured as follows. 
%
Section \ref{sec:related-work} reviews related work on recommendation. 
%
We define the problem that we need to tackle in Section \ref{sec:problem}.
%
Section \ref{sec:framework} discusses about the framework we use (i.e. GraphX), and we describe the algorithms we use in GraphX in Section \ref{sec:algo}.
%
In Section \ref{sec:insights}, we analyze some of the advantages and disadvantages of GraphX and discuss what we have learned while exploring the framework.
%
We present our experimental results in Section \ref{sec:exp}.
%
Section \ref{sec:conclusion} states some future work of this project and concludes the paper.
